% The document class supplies options to control rendering of some standard
% features in the result.  The goal is for uniform style, so some attention 
% to detail is *vital* with all fields.  Each field (i.e., text inside the
% curly braces below, so the MEng text inside {MEng} for instance) should 
% take into account the following:
%
% - author name       should be formatted as "FirstName LastName"
%   (not "Initial LastName" for example),
% - supervisor name   should be formatted as "Title FirstName LastName"
%   (where Title is "Dr." or "Prof." for example),
% - degree programme  should be "BSc", "MEng", "MSci", "MSc" or "PhD",
% - dissertation title should be correctly capitalised (plus you can have
%   an optional sub-title if appropriate, or leave this field blank),

% - year              should be formatted as a 4-digit year of submission
%   (so 2023 rather than the accademic year, say 2022/23 say).
%
% Note there is a *strict* requirement for the poster to be in portrait 
% format so that we display them on the poster boards available.

\documentclass[ % the name of the author
                author={Important Person},
                % the name of the supervisor
                supervisor={Academic Guide},
                % the degree programme
                    degree={MEng},
                % the dissertation    title (which cannot be blank)
                     title={Some Structural Guidelines for CS Final Year Posters},
                % the dissertation subtitle (which can    be blank)
                  subtitle={},
                % the dissertation     type
                      type={enterprise},
                % the year of submission
                      year={2023} ]{poster}

\begin{document}

% -----------------------------------------------------------------------------

  \begin{frame}{} 
  
    \vfill
    
    \begin{columns}[t]
      \begin{column}{0.900\linewidth}
        \begin{block}{\Large Introduction}
          It is hard to give generic advice about what form your poster should take, since each project relates to a different topic and each student will be at a different stage wrt. completeness.
          Therefore, the best approach is to focus on the underlying aim of the poster presentation: essentially the intention is for you to get early, objective opinions about your work and then (ideally) improve it as a result.
     
          With this in mind, one idea is to:
          \begin{enumerate}
            \item think about how to explain your project to someone, and questions you might want an answer to or opinion on,
            \item consider the poster as a set of slides, which support an elevator        pitch\footnote{\url{http://en.wikipedia.org/wiki/Elevator_pitch}} for either the technical and/or business plan part, then
            \item focus the poster content on the part you feel you need the most input on.
          \end{enumerate}
        
          \noindent
          Another approach is to adopt standard advice about developing research       posters\footnote{\url{https://colinpurrington.com/tips/poster-design/}}, then produce a stand-alone result that summarises your project (see examples on walls throughout the MVB).
          Either way, the blocks below attempt to outline some potential examples of content, but note you need not stick to them
        \end{block}
      \end{column}
    \end{columns}
    
    \vfill
    
    \begin{columns}[t]
      \begin{column}{0.422\linewidth}
        \begin{block}{\Large 1. Project Introduction}
          In this section you want to sell the concept of the project to the reader.
          You can think of this as answering the questions below:
          \begin{itemize}
            \item Why is the project interesting/useful?
            \item What has come before it?
            \item How is this project different from what has already been done?
            \item What is the end goal of the project?
          \end{itemize}
        \end{block}
      \end{column}
      
      \begin{column}{0.422\linewidth}
        \begin{block}{\Large 2. Main Problem/Deliverable}
          Here you may describe the main problem that you are trying to solve in more detail, or the deliverable that you will be working to deliver.
          You can do this by:
          \begin{itemize}
            \item Showing a figure of the method
            \item Describing the problem setup/specification
            \item Showing figure(s) of designs
          \end{itemize}
        \end{block}
      \end{column}
    \end{columns}
    
    \vfill
    
    \begin{columns}[t]
      \begin{column}{0.422\linewidth}
        \begin{block}{\Large 3. Preliminary Results}
          Preliminary results could be in the form of a 
          \begin{itemize}
            \item table, use \url{https://www.tablesgenerator.com/} if you're not familiar with LaTeX's table formatting.
            \item figures, such as line/bar/scatter plots - remember captions!
            \item qualitative results showing examples of what you have achieved so far.
      
          \end{itemize}
        \end{block}
      \end{column}
      \begin{column}{0.422\linewidth}
        \begin{block}{\Large 4. Progress and Status}
          Example content might include:
        
          \begin{itemize}
            \item a list of complete and incomplete aims and objectives,
            \item a list of open questions or problems,
                  and
            \item your plan for completing the project, inc. required deliverables.
          \end{itemize}
        \end{block}
      \end{column}
    \end{columns}
    
    \vfill
  
  \end{frame}

% -----------------------------------------------------------------------------

\end{document}



